\documentclass[onecolumn, draftclsnofoot,10pt, compsoc]{IEEEtran}
\usepackage{graphicx}
\usepackage{url}
\usepackage{setspace}
\usepackage{geometry}
\usepackage{tabularx}
\geometry{textheight=9.5in, textwidth=7in}

\def \CapstoneTeamName{		    Comedy Robot Team}
\def \CapstoneTeamNumber{		43}
\def \GroupMemberOne{			Timothy Bui}
\def \GroupMemberTwo{			Yuhang (Tony) Chen}
\def \GroupMemberThree{			Brian Ozarowicz}
\def \GroupMemberFour{			Trevor Webster}
\def \CapstoneProjectName{		Building More Self-Aware\linebreak Everyday Robots}
\def \CapstoneSponsorCompany{	SHARE Lab}
\def \CapstoneSponsorPerson{	Dr. Naomi Fitter}

% Uncomment the appropriate line below so that the document type works
\def \DocType{	Problem Statement
				%Requirements Document
				%Technology Review
				%Design Document
				%Progress Report
				}
			
\newcommand{\NameSigPair}[1]{\par
\makebox[2.75in][r]{#1} \hfil 	\makebox[3.25in]{\makebox[2.25in]{\hrulefill} \hfill		\makebox[.75in]{\hrulefill}}
\par\vspace{-12pt} \textit{\tiny\noindent
\makebox[2.75in]{} \hfil		\makebox[3.25in]{\makebox[2.25in][r]{Signature} \hfill	\makebox[.75in][r]{Date}}}}
% If the document is not to be signed, uncomment the RENEWcommand below
\renewcommand{\NameSigPair}[1]{#1}

%%%%%%%%%%%%%%%%%%%%%%%%%%%%%%%%%%%%%%%
\begin{document}
\begin{titlepage}
    \pagenumbering{gobble}
    \begin{singlespace}
    	\includegraphics[height=4cm]{coe_v_spot1}
        \hfill 
        % If you have a logo, use this includegraphics command to put it on the coversheet.
        %\includegraphics[height=4cm]{CompanyLogo}   
        \par\vspace{.2in}
        \centering
        \scshape{
            \huge CS Capstone \DocType \par
            %{\large\today}\par
            {\large Fall 2019}\par
            \vspace{.5in}
            \textbf{\Huge\CapstoneProjectName}\par
            \vfill
            {\large Prepared for}\par
            %\Huge \CapstoneSponsorCompany\par
            %\vspace{5pt}
            {\Large\NameSigPair{\CapstoneSponsorPerson}\par}
            {\large Prepared by }\par
            Group\CapstoneTeamNumber\par
            % comment out the line below this one if you do not wish to name your team
            \CapstoneTeamName\par 
            \vspace{5pt}
            {\Large
                \NameSigPair{\GroupMemberOne}\par
                \NameSigPair{\GroupMemberTwo}\par
                \NameSigPair{\GroupMemberThree}\par
                \NameSigPair{\GroupMemberFour}\par
            }
            \vspace{20pt}
        }
        \begin{abstract}
        	\noindent As human-robot interactions become increasingly commonplace it is important to make the experience feel natural. By studying the ability of a robotic comedian to perceive social cues from an audience and adjust its behavior appropriately insight can be gained into how interpersonal communications between robots and humans can be improved. Our project will study the ability of a robot to determine how well a joke is received based on the laughter it hears and seek ways to improve the success rate of its analysis.
        \end{abstract}
    \end{singlespace}
\end{titlepage}
\newpage
\pagenumbering{arabic}
\tableofcontents
% uncomment this (if applicable). Consider adding a page break.
%\listoffigures
%\listoftables
\clearpage

\section{Problem}
Interactions with robots have become increasingly commonplace in modern society as new technologies and uses for robots emerge in fields ranging from industry to rehabilitation to everyday personal assistants. One of the major problems present in human-robot interactions is a robot's ability to understand and adapt to human reactions, both verbal and nonverbal, to changes in a situation. This is a complex problem which includes a large number of different types of research such as human recognition, natural language processing, predictive analysis, and task planning.\par
\vspace{.4cm}
\noindent Currently, many human-robot interactions feel unnatural or ineffective as users are required to give clear, specific commands to robots in order to get them to perform as intended. In many situations robots are not sophisticated enough to react naturally in real environments as they encounter events that are outside the scope of their programming.

\section{Proposed Solution}
This project aims to improve the ability of robots to detect and respond to humans. The medium that was chosen to accomplish this is stand-up comedy. According to our client, Dr. Naomi Fitter, `'comedy performance is one setting in which monitoring one's own state and the responses of nearby people is important for interaction success'`. Stand-up comedy in particular provides an environment where the comedian can perform a joke and then analyze the audience's reactions before telling their next joke, making adjustments to their future behavior if needed in order to improve the connection being made with their audience. This makes it an ideal testing environment for helping improve a robot's ability to respond accurately to laughter.\par
\vspace{.4cm}
\noindent This work has implications for improving human-robot interactive experiences in many contexts, such as improving how natural it feels to use AI assistants and how accurately they can respond to inquiries.

\section{Performance Metrics}
The primary performance metric for this project will be to compare the robot's ability to evaluate the successfulness of a joke to the results from human-performed evaluations. By analyzing its own past performances the robot will create a classifier that sorts jokes into one of three categories: a hit, a bomb, or a neutral response. Our goal is for the robot to be able to classify the successfulness of a joke with at least 85\% accuracy, matching the success rate observed when humans do the classification.

\end{document}