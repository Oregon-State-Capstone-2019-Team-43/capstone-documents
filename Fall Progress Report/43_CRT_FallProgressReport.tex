\documentclass[onecolumn, draftclsnofoot,10pt, compsoc]{IEEEtran}
\usepackage{graphicx}
\usepackage{url}
\usepackage{setspace}
\usepackage{geometry}
\usepackage{tabularx}
\geometry{textheight=9.5in, textwidth=7in}

\def \CapstoneTeamName{		    Comedy Robot Team}
\def \CapstoneTeamNumber{		43}
\def \GroupMemberOne{			Timothy Bui}
\def \GroupMemberTwo{			Yuhang (Tony) Chen}
\def \GroupMemberThree{			Brian Ozarowicz}
\def \GroupMemberFour{			Trevor Webster}
\def \CapstoneProjectName{		Building More Self-Aware\linebreak Everyday Robots}
\def \CapstoneSponsorCompany{	SHARE Lab}
\def \CapstoneSponsorPerson{	Dr. Naomi Fitter}

% Uncomment the appropriate line below so that the document type works
\def \DocType{	%Problem Statement
				%Requirements Document
				%Technology Review
				%Design Document
				Progress Report
				}
			
\newcommand{\NameSigPair}[1]{\par
\makebox[2.75in][r]{#1} \hfil 	\makebox[3.25in]{\makebox[2.25in]{\hrulefill} \hfill		\makebox[.75in]{\hrulefill}}
\par\vspace{-12pt} \textit{\tiny\noindent
\makebox[2.75in]{} \hfil		\makebox[3.25in]{\makebox[2.25in][r]{Signature} \hfill	\makebox[.75in][r]{Date}}}}
% If the document is not to be signed, uncomment the RENEWcommand below
\renewcommand{\NameSigPair}[1]{#1}

%%%%%%%%%%%%%%%%%%%%%%%%%%%%%%%%%%%%%%%
\begin{document}
\begin{titlepage}
    \pagenumbering{gobble}
    \begin{singlespace}
    	\includegraphics[height=4cm]{coe_v_spot1}
        \hfill 
        % If you have a logo, use this includegraphics command to put it on the coversheet.
        %\includegraphics[height=4cm]{CompanyLogo}   
        \par\vspace{.2in}
        \centering
        \scshape{
            \huge CS Capstone \DocType \par
            %{\large\today}\par
            {\large Fall 2019}\par
            \vspace{.5in}
            \textbf{\Huge\CapstoneProjectName}\par
            \vfill
            {\large Prepared for}\par
            %\Huge \CapstoneSponsorCompany\par
            %\vspace{5pt}
            {\Large\NameSigPair{\CapstoneSponsorPerson}\par}
            {\large Prepared by }\par
            Group\CapstoneTeamNumber\par
            % comment out the line below this one if you do not wish to name your team
            \CapstoneTeamName\par 
            \vspace{5pt}
            {\Large
                \NameSigPair{\GroupMemberOne}\par
                \NameSigPair{\GroupMemberTwo}\par
                \NameSigPair{\GroupMemberThree}\par
                \NameSigPair{\GroupMemberFour}\par
            }
            \vspace{20pt}
        }
        \begin{abstract}
        	\noindent This document provides a summary of the progress made during Fall term on the 'Building More Self-Aware Everyday Robots' Capstone project. It examines the research and planning work done so far in relation to the established project goals and gives the plan for what will be done over the break in preparation for proceeding to implementation.
        \end{abstract}
    \end{singlespace}
\end{titlepage}
\newpage
\pagenumbering{arabic}
\tableofcontents
% uncomment this (if applicable). Consider adding a page break.
%\listoffigures
%\listoftables
\clearpage

\section{Purpose and Goals}
The purpose of our project is to study how human-robot interactions can be improved by observing the ability of a robotic stand-up comedian to read and respond to cues from its audience. Our goals are to use machine learning on a dataset of recordings from previous comedy performances to train models for detecting laughter and using that response to decide whether a joke was a hit or a bomb.\par
\vspace{.3cm}
\noindent This analysis can be used by the robot to make real-time adjustments to its performance based on the perceived audience preferences in order to improve the success of their interpersonal communications. The work also has potential applications outside the field of comedy in improving the experience of interactions with AI assistants and other autonomous systems.

\section{Current Status}
We have met with our client several times over the course of the term to discuss the project scope, deliverable goals, and details of implementation. We have done background reading on the origin of the project and the robot team's previous work and identified where our capstone project will build on their work to advance the overall project. Our client has provided the dataset of recordings and joke classifications we will be working with and we have reviewed the files to become familiar with the data documentation and labeling. We have conducted research into various methods of machine learning and noise suppression that could potentially be used for our work and examined the specs of the Nao V6 robot to determine what can be accomplished with the provided hardware. We have completed all project planning and documentation assigned for the term and the documents were sent to our client for review and verification to proceed to implementation next term, which has been granted.

\section{Problems and Solutions}
To implement the machine learning we need to learn to use the SciKit library in Python, requiring familiarity with other libraries including Numpy, Panda, and Matplotlib. We have purchased a Udemy online class to learn and experiment with those libraries over break.\par
\vspace{.3cm}
\noindent A lot of machine learning knowledge is required, including statistics and math concepts we need to learn such as logistic regression and curve fitting. Three of us are taking ST421 this term, which connects to much of the related knowledge. The Udemy class also has a few lessons about it.\par
\vspace{.3cm}
\noindent The Nao robot's hardware places limits what level of processing can be accomplished in real time, particularly for noise suppression. This is somewhat processor intensive and may not be able to run well onboard the robot. We are searching for lightweight applications to use in performing the audio processing to prevent overloading the robot's capabilities.

\section{Weekly Progress Summary}
Week 1:
\begin{itemize}
\item Met group members
\item Established team communication system and standards
\item Conducted initial research into project background
\end{itemize}

Week 2:
\begin{itemize}
\item Contacted client to make introductions
\item Wrote individual Problem Statements
\item Met with client to discuss project details and definite goals and breakdown of the specific tasks involved
\item Attended a performance by the robot to observe the project in action
\end{itemize}

Week 3:
\begin{itemize}
\item Received dataset from client
\item Received project documentation and research paper from client
\item Wrote first draft of the Requirements Document and sent to client for review
\item Received email from client with additional details on project scope and desired deliverables
\item Wrote the group Problem Statement
\end{itemize}

Week 4:
\begin{itemize}
\item Continued background reading on the project and related research from client's resources
\item Finished the Requirements Document
\end{itemize}

Week 5:
\begin{itemize}
\item Researched technology options for various tasks involved in the project
\item Wrote first drafts of individual Tech Reviews recommending initial approaches to the tasks
\item Met with client to detail task breakdowns and their applications to the project goals
\end{itemize}

Week 6:
\begin{itemize}
\item Received email from client with additional sources for research into past studies of project-related concepts
\item Finished the Tech Reviews
\item Met with client to further discuss project deliverables
\end{itemize}

Week 7:
\begin{itemize}
\item Wrote first draft of the Design Document
\item Met with client to discuss plans for preparation of initial implementation
\end{itemize}

Week 8:
\begin{itemize}
\item Finished the Design Document
\item Sent the Design Document to client for review
\item Began research into application of the methodologies selected in our Tech Reviews
\end{itemize}

Week 9:
\begin{itemize}
\item Continued methodology research and implementation preparation
\end{itemize}

Week 10:
\begin{itemize}
\item Provided project documentation to client for review and secured approval of work done so far
\item Held final meeting with client to discuss research and preparations to be done over break to make ready for proceeding to implementation next term
\item Wrote the Fall Term Progress Report
\end{itemize}

\section{Retrospective}
\vspace{.4cm}
\begin{tabular}{|p{0.04\linewidth}|p{0.3\linewidth}|p{0.3\linewidth}|p{0.3\linewidth}|}
\hline
\centering Task* &
\centering Positives &
\centering Deltas &   
\centering Actions
\tabularnewline
\hline
1  & 
Read documents provided by client for background and attended a performance by the robot  & 
Hard to find past work on laughter detection in audio, most studies are for voice detection or music  & 
Conduct ongoing research into related work in audio analysis 
\tabularnewline
\hline
2  & 
There is a good amount of data to work with  & 
Most of the data is from the previous version of the Nao robot  & 
Work with the V5 data first to establish a functional model then proceed to use the V6 data 
\tabularnewline
\hline
3  & 
Received guidance from client on which options were most appropriate for this usage  & 
Must identify which learning method is best for our data, keeping in mind the requirement that it must be able to run in real-time in the end result of our work  & 
Run the audio analysis using each learning method and compare the results to determine the most successful application
\tabularnewline
\hline
4  & 
Looked over the robot team's current implementation  & 
May need to make room for additional preprocessing such as noise suppression  & 
Replicate the current implementation to produce a working base system then introduce noise suppression functionality and gauge any affect on performance 
\tabularnewline
\hline
5  & 
Purchased Udemy course to get up to speed on machine learning  & 
Must familiarize with the learning process and how new methods can be implemented and tested  & 
Conduct research over break into different learning methods and read about implementation with SciKit 
\tabularnewline
\hline
6  & 
Python library was found which can be used for our needs and an initial implementation was prepared  & 
The library may prove to be too processor intensive for real-time usage  & 
Observe processor requirements of the library in use and modify to be more lightweight if needed 
\tabularnewline
\hline
\end{tabular}
\vspace{.4cm}
\par
\noindent * Task List for Fall Term:\par
\noindent Task 1: Familiarize with project background\par
\noindent Task 2: Examine the dataset of recordings and documentation\par
\noindent Task 3: Research machine learning options\par
\noindent Task 4: Set up for interfacing with the dataset\par
\noindent Task 5: Prepare for machine learning implementation\par
\noindent Task 6: Prepare for noise suppression implementation

\end{document}